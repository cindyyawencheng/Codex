\documentclass[12pt]{article} 

%%%%%%%%%%%%%%%%%%%%%%%%%%%%%%%%%%%%%%%%%%%%%%%%%%
%%%%%%%%%%%%%%%%%%%% PREAMBLE %%%%%%%%%%%%%%%%%%%%
%%%%%%%%%%%%%%%%%%%%%%%%%%%%%%%%%%%%%%%%%%%%%%%%%%


% -------------------- defaults -------------------- %
% load lots o' packages

% approx iid
\newcommand\simiid{\stackrel{\mathclap{\normalfont\mbox{\tiny{iid}}}}{\sim}}
\usepackage{natbib}
\usepackage{titlecaps}
\Addlcwords{a about above after again against all am an and any are aren't as at be because been before being below between both but by can't cannot could couldn't did didn't do does doesn't doing don't down during each few for from further had hadn't has hasn't have haven't having he he'd he'll he's her here here's hers herself him himself his how how's i i'd i'll i'm i've if in into is isn't it it's its itself let's me more most mustn't my myself no nor not of off on once only or other ought our ours ourselves out over own same shan't she she'd she'll she's should shouldn't so some such than that that's the their theirs them themselves then there there's these they they'd they'll they're they've this those through to too under until up very was wasn't we we'd we'll we're we've were weren't what what's when when's where where's which while who who's whom why why's with won't would wouldn't you you'd you'll you're you've your yours yourself yourselves}

% Fonts
% at some point figure out bolding ...
\usepackage[default,osfigures,scale=0.95]{opensans}
%\usepackage[sfdefault,scaled=.85]{FiraSans}
\usepackage[T1]{fontenc}
\usepackage{textcomp}
\usepackage[varqu,varl]{zi4}% inconsolata typewriter
\usepackage{amsmath,amsthm}
\usepackage[cmintegrals]{newtxsf}
\usepackage[T1]{fontenc}
\usepackage{ae}

% layout control
\usepackage[paper=a4paper,left=25mm,right=25mm,top=20mm,bottom=25mm]{geometry}
\usepackage[onehalfspacing]{setspace}
\setlength{\parskip}{.5em}
\usepackage{rotating}
\usepackage{setspace}
\usepackage{fancyhdr}
\usepackage{parallel}
\usepackage{parcolumns}
\usepackage{pdflscape}
% math typesetting
\usepackage{array}
\usepackage{amsmath}
\usepackage{amssymb}
\usepackage{amsfonts}

% tables
\usepackage{tabularx}
\usepackage{booktabs}
\usepackage{multicol}
\usepackage{multirow}
\usepackage{longtable}
\usepackage{tabularx}
\usepackage{etoolbox}
\usepackage{booktabs}

\usepackage[%
decimalsymbol=.,
digitsep=fullstop
]{siunitx}

% to adapt caption style
\usepackage[font={small},labelfont=bf]{caption}

% footnotes at bottom
\usepackage[bottom]{footmisc}

% to change enumeration symbols begin{enumerate}[(a)]
\usepackage{enumerate}

% to make enumerations and itemizations within paragraphs or
% lines. f.i. begin{inparaenum} for (a) is (b) and (c)
\usepackage{paralist}

% to colorize links in document. See color specification below
\usepackage[x11names]{xcolor}

% for multiple references and insertion of the word "figure" or "table"
% \usepackage{cleveref}

% load the hyper-references package and set document info
\usepackage[pdftex]{hyperref}

% graphics stuff
\usepackage{subfig}
\usepackage{graphicx}
\usepackage[space]{grffile} % allows us to specify directories that have spaces
\usepackage[section]{placeins} % prevents floats from moving past a \FloatBarrier or section
\usepackage{tikz}
\usepackage{caption}

% \usepackage{pgfplots}

% define clickable links and their colors
\hypersetup{
	unicode=false,          % non-Latin characters in Acrobat's bookmarks
	pdftoolbar=true,        % show Acrobat's toolbar?
	pdfmenubar=true,        % show Acrobat's menu?
	pdffitwindow=false,     % window fit to page when opened
	pdfstartview={FitH},    % fits the width of the page to the window
	pdfnewwindow=true,%
	pdfauthor={Cindy Cheng and Shahryar Minhas},%
	pdftitle={Aid something},%
	colorlinks,%
	citecolor=black,%
	filecolor=black,%
	linkcolor=black,%
	urlcolor=RoyalBlue4%
	}

% Including External Code
\usepackage{verbatim}
\usepackage{listings}
\lstset{
	language=R,
	basicstyle=\scriptsize\ttfamily,
	commentstyle=\ttfamily\color{gray},
	numbers=left,
	numberstyle=\ttfamily\color{gray}\footnotesize,
	stepnumber=1,
	numbersep=5pt,
	backgroundcolor=\color{white},
	showspaces=false,
	showstringspaces=false,
	showtabs=false,
	frame=single,
	tabsize=2,
	captionpos=b,
	breaklines=true,
	breakatwhitespace=false,
	title=\lstname,
	escapeinside={},
	keywordstyle={},
	morekeywords={}
	}


% -------------------------------------------------- %


% -------------------- title -------------------- %

\title{Do Too Many Cooks Spoil the Broth? The Effect of Increasing Participation on the Codex Food Standard-Making Process
\\ Preliminary draft: Please do not cite
}
\author{	Cindy Cheng \\
	\texttt{cindy.cheng@hfp.tum.de}
	\and
	Sebastian Klotz \\
	\texttt{Sebastian.Klotz@wti.org}
}
\date{\today}


% \setlength{\headheight}{15pt}
% \setlength{\headsep}{20pt}
% \pagestyle{fancyplain}
 
% \fancyhf{}
 
% \lhead{\fancyplain{}{}}
% \chead{\fancyplain{}{}}
% \rhead{\fancyplain{}{}}
% \rfoot{\fancyplain{}{}}

% ----------------------------------------------- %


% -------------------- customizations -------------------- %

%\graphicspath{{~/Users/cindycheng/Dropbox/ForeignAid/graphics/}}
\graphicspath{{graphics/}}
\makeatletter
\def\input@path{{graphics/}}
 \makeatother

% \makeatletter
%\def\input@path{{/Users/janus829/Dropbox/Research/ForeignAid/Graphics/}}
% \makeatother
%\graphicspath{{/Users/janus829/Dropbox/Research/ForeignAid/Graphics/}}

% easy commands for number propers
\newcommand{\first}{$1^{\text{st}}$}
\newcommand{\second}{$2^{\text{nd}}$}
\newcommand{\third}{$3^{\text{rd}}$}
\newcommand{\nth}[1]{${#1}^{\text{th}}$}

% square bracket matrices
\let\bbordermatrix\bordermatrix
\patchcmd{\bbordermatrix}{8.75}{4.75}{}{}
\patchcmd{\bbordermatrix}{\left(}{\left[}{}{}
\patchcmd{\bbordermatrix}{\right)}{\right]}{}{}

% Extra functions
\newcommand{\bl}[1]{{\mathbf #1}}
\newcommand{\bs}[1]{{\boldsymbol #1}}
\newcommand{\etr}{{\rm etr}}
\newcommand{\tr}{\text{tr}}
\def\checkmark{\tikz\fill[scale=0.4](0,.35) -- (.25,0) -- (1,.7) -- (.25,.15) -- cycle;}

\newcommand{\Exp}[1]{{\text{E}}[ \ensuremath{ #1 } ]  }
\newcommand{\Var}[1]{{\text{Var}}[ \ensuremath{ #1 } ]  }
\newcommand{\Cov}[1]{{\text{Cov}}[ \ensuremath{ #1 } ]  }
\newcommand{\Cor}[1]{{\text{Cor}}[ \ensuremath{ #1 } ]  }


%% Break urls at hyphens
\PassOptionsToPackage{hyphens}{url}\usepackage{hyperref}
%% Define new column type so that theme model results fit in a column
\usepackage{array}
\newcolumntype{R}[1]{>{\raggedleft\let\newline\\\arraybackslash\hspace{0pt}}m{#1}}

% Add some colors
\definecolor{green1}{RGB}{229,245,224}
\definecolor{green2}{RGB}{161,217,155}
\definecolor{green3}{RGB}{49,163,84}

\definecolor{blue1}{RGB}{222,235,247}
\definecolor{blue2}{RGB}{158,202,225}
\definecolor{blue3}{RGB}{49,130,189}

% easy command for boldface math symbols
\newcommand{\mbs}[1]{\boldsymbol{#1}}

% define bibliography style




% -------------------------------------------------------- %


%%%%%%%%%%%%%%%%%%%%%%%%%%%%%%%%%%%%%%%%%%%%%%%%%%
%%%%%%%%%%%%%%%%%%%% DOCUMENT %%%%%%%%%%%%%%%%%%%%
%%%%%%%%%%%%%%%%%%%%%%%%%%%%%%%%%%%%%%%%%%%%%%%%%%
\doublespacing 

\begin{document}

\maketitle

\begin{abstract}

\singlespacing{


}
\end{abstract}

\newpage 

%%%%%%%% INTRO %%%%%%%%
\section{Introduction}


A defining feature of the post World War II era has been the increasing democratization of the types of actors that participate in the governance of international political and economic affairs. For one, the process of decolonization has meant that an increasing number of countries have not only become self governing but have become better able to represent their interests on the international stage. For another, the confluence of rising economic development and greater economic openness across the globe has meant that more countries have both interest and ability in making their voice heard in global governance issues. New and developing countries are not the only new entrants to the global stage. Private and transnational actors are making their interests heard a variety of areas that affect both the state and market, from international organizations to standard setting organizations. 

In short, a variety of contexts, both new and previously marginalized actors are developing a larger voice and more representation in global governing tables. While on the one hand, such increased participation may appeal to democratic norms of representation and legitimacy, they also raise serious questions about the efficiency of the resultant governance outcomes. Such issues echo how the governance outcomes of autocratic and democratic regimes are often compared. In terms of outcomes, the tradeoff is framed as one between autocratic efficiency and democratic sluggishness while in terms of process, the tradeoff is characterized as being between autocractic tyranny and democratic legitimacy. 

We investigate this question in the specific case of the Codex Alimentarius Commission (Codex). The Codex is an international food standard organization founded by the FAO and WHO in 1963 and became the world's \textit{de facto} reference point for food standards with the creation of the WTO in 1995. In a sense the Codex represents a `hard' case to investigate the tension between process and outcomes in that i) the criteria for formal participation is well defined in terms of the nation state ii) the policy outcomes are well defined in terms of food safety standards and iii) the process through which policy is formulated is well institutionalized both in terms of the bodies in which participation can take place and in terms of the process through which proposals for standards reach fruition. Moreover, the change in the Codex's status from an organization that was ultimately limited to issuing non-binding food safety standards to one that enjoyed the formal backing of the WTO dispute settlement mechanism allows us to investigate how formal institutionalization of power affects the relationship between participation and governance outcomes. 
%%%%%%%%%%%%%%%%%%%%%%%


%%%%% Theory %%%%%


\section{Democratic Legitimacy and Governance Outcomes}


\citet{agne_etal:2015} argue that the normative ideal of greater stakeholder participation does not necessarily translate into practicable democratic outcomes. More likely than not, greater participation leads to greater stalemate in the policy making process.


A number of works also raise the importance of not just the role of increased individual actor participation in governing bodies, but how the possibility of different coalitions of actors can affect the governance process. \citet{perezDuran:2018} for instance identifies three interest groups that can exert influence in the decisionmaking bodies of European Union agencies: representatives of capital (e.g. industry), representatives of labor (e.g. trade unions) and citizen's representatives (e.g. non-governmental organizations, consumer groups). Meanwhile \citet{goldbach:2015} for instance, argues that politicians and firms form an important coaliation for influencing international banking regulation.  





% number of delegations/delegates
%    H1: more delegations/delegates, 
%      H1A: longer time for standard development, 
%      H1B: fewer standards standards developed
%      H1C: higher likelihood of amendment/revision/revocation 


%  composition of delegations/delegates
%    developing/developed countries
  
%    H2: greater participation  of developing countries (delegates/chair) .....  

%    ngo/private/igo/countries

%    H3: greater participation of non-state actors (ngo/private) ....

%    something about representation?

%    H4: regional coordinating committees.....

%      when is there consensus within regional coordinating commmitteees/

%      when is there in-fighting 


%  WTO creation
  
%    After WTO creation, much more important to make sure standards reflect actor's interests

%        ... which leads to more political stalement and disagreement 
%        H4: standards take longer to develop/fewer standards after WTO created
  
%        ... which leads to more political horsetrading and compromise
%        H5: more standards after WTO created

%        which pathway depends on how many standards an actors has a vested stake in
%        operationalize this by number of committees an actors is invovled in weighted by delegatization size


% # H6: CCEXEC; something about agenda-setting? 


% # precautionary principal (EU) vs. scientific principle (US)
%   # where does the scientific evidence come from?
%   # JECFA vs EFSA

%%%%%%%%%%%%%%%%%%%%%%


%%%%% Description of Codex %%%%%

\section{Codex}

The Codex Alimentarius Commission is an ideal place to investigate these issues. 



\citet{veggeland:2005} have found that participants changed their behavior in response to Codex's elevation by the WTO as the ``central refernce point for the elaboration of international food standards.''



%%%%%%%%%%%%%%%%%%%%%%

%%%%% Data %%%%%



\section{Data}
The length of time that it takes for a proposal to become adopted is interpreted as the amount of insurmountable disagreement. 


\subsection{Control Variables}


Meeting type:  \citet{victor:1997} notes that Comodity committess are generally dominanted by government and indusry representatives. Meanwhile horizontal committees have more firewalls to ``protect the independence of the expert advisory process''.
%%%%%%%%%%%%%%%%%%%%%%


%%%%% Empirics %%%%%
\input{Paper/5_codexEmpirics}
%%%%%%%%%%%%%%%%%%%%%%

%%%%% Conclusion %%%%%
\section*{Conclusion}


Future work includes gathering more fine-grained data on the individual delegates to fully test Halliday et al. (2013)'s arguments. 


In future work, we will explore the extent to which increasing participation affects not only the efficiency of the policy process, but the representativeness of the resulting policy. In the case of Codex, this means investigating the extent to which a certain standard is created for a narrow or broader range of commodities.  

Future work is also needed to explore the extent to which Codex standards are actually being adopted and implemented by member states, what explains variation in adoption and how adoption when it happens affects trade and food safety outcomes. 
%%%%%%%%%%%%%%%%%%%%%%


\newpage
\bibliographystyle{chicago}
\bibliography{Paper/codex.bib}

 

\end{document} 
