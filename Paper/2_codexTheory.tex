

\section{Democratic Legitimacy and Governance Outcomes}


\citet{agne_etal:2015} argue that the normative ideal of greater stakeholder participation does not necessarily translate into practicable democratic outcomes. More likely than not, greater participation leads to greater stalemate in the policy making process.


A number of works also raise the importance of not just the role of increased individual actor participation in governing bodies, but how the possibility of different coalitions of actors can affect the governance process. \citet{perezDuran:2018} for instance identifies three interest groups that can exert influence in the decisionmaking bodies of European Union agencies: representatives of capital (e.g. industry), representatives of labor (e.g. trade unions) and citizen's representatives (e.g. non-governmental organizations, consumer groups). Meanwhile \citet{goldbach:2015} for instance, argues that politicians and firms form an important coaliation for influencing international banking regulation.  





% number of delegations/delegates
%    H1: more delegations/delegates, 
%      H1A: longer time for standard development, 
%      H1B: fewer standards standards developed
%      H1C: higher likelihood of amendment/revision/revocation 


%  composition of delegations/delegates
%    developing/developed countries
  
%    H2: greater participation  of developing countries (delegates/chair) .....  

%    ngo/private/igo/countries

%    H3: greater participation of non-state actors (ngo/private) ....

%    something about representation?

%    H4: regional coordinating committees.....

%      when is there consensus within regional coordinating commmitteees/

%      when is there in-fighting 


%  WTO creation
  
%    After WTO creation, much more important to make sure standards reflect actor's interests

%        ... which leads to more political stalement and disagreement 
%        H4: standards take longer to develop/fewer standards after WTO created
  
%        ... which leads to more political horsetrading and compromise
%        H5: more standards after WTO created

%        which pathway depends on how many standards an actors has a vested stake in
%        operationalize this by number of committees an actors is invovled in weighted by delegatization size


% # H6: CCEXEC; something about agenda-setting? 


% # precautionary principal (EU) vs. scientific principle (US)
%   # where does the scientific evidence come from?
%   # JECFA vs EFSA
