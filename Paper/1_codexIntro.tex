\section{Introduction}


A defining feature of the post World War II era has been the increasing democratization of the types of actors that participate in the governance of international political and economic affairs. For one, the process of decolonization has meant that an increasing number of countries have not only become self governing but have become better able to represent their interests on the international stage. For another, the confluence of rising economic development and greater economic openness across the globe has meant that more countries have both interest and ability in making their voice heard in global governance issues. New and developing countries are not the only new entrants to the global stage. Private and transnational actors are making their interests heard a variety of areas that affect both the state and market, from international organizations to standard setting organizations. 

In short, a variety of contexts, both new and previously marginalized actors are developing a larger voice and more representation in global governing tables. While on the one hand, such increased participation may appeal to democratic norms of representation and legitimacy, they also raise serious questions about the efficiency of the resultant governance outcomes. Such issues echo how the governance outcomes of autocratic and democratic regimes are often compared. In terms of outcomes, the tradeoff is framed as one between autocratic efficiency and democratic sluggishness while in terms of process, the tradeoff is characterized as being between autocractic tyranny and democratic legitimacy. 

We investigate this question in the specific case of the Codex Alimentarius Commission (Codex). The Codex is an international food standard organization founded by the FAO and WHO in 1963 and became the world's \textit{de facto} reference point for food standards with the creation of the WTO in 1995. In a sense the Codex represents a `hard' case to investigate the tension between process and outcomes in that i) the criteria for formal participation is well defined in terms of the nation state ii) the policy outcomes are well defined in terms of food safety standards and iii) the process through which policy is formulated is well institutionalized both in terms of the bodies in which participation can take place and in terms of the process through which proposals for standards reach fruition. Moreover, the change in the Codex's status from an organization that was ultimately limited to issuing non-binding food safety standards to one that enjoyed the formal backing of the WTO dispute settlement mechanism allows us to investigate how formal institutionalization of power affects the relationship between participation and governance outcomes. 